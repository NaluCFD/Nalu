\documentclass[letterpaper, 11pt]{article}
\usepackage[margin=1in]{geometry}
\usepackage{graphicx}
\usepackage{lipsum}
\usepackage{lmodern}
%\usepackage{tgadventor}
\usepackage{fancyhdr}
\usepackage{enumitem}
\usepackage{url}


% Header
\pagestyle{fancy}
\fancyhf{}
\lhead{\textbf{ME469/CME369: Computational Methods in Fluid Mechanics}}
%\rhead{\textbf{Spring 2024}}
%\rhead{\includegraphics[height=0.5in]{stanford_logo.png}}

% Title
\title{\vspace{-1.5cm}\textbf{Accessing the course cluster, me469-cluster@stanford.edu}}
\author{Stefan Domino}
%\date{\today}

\begin{document}
\maketitle
\thispagestyle{fancy}

% Itemized list
\noindent
To connect to the cluster from your local machine:
\begin{itemize}[noitemsep]
  \item Connect to the Stanford VPN: \url{https://uit.stanford.edu/service/vpn} where you can download the Mac of Windows VPN client.
  \item Use SSH to access the cluster: 

      \begin{center}
      \texttt{ssh sunetid@me469-cluster.stanford.edu}
      \end{center}

      using your SUNet name and password. This step requires an installation of Stanford's VPN software.
\end{itemize}

\noindent
Once you are logged in, you can submit jobs to the queue using the Slurm 
scheduling manager. We provide a template job submission file that you can 
modify and reuse for all your simulations.
\begin{itemize}[noitemsep]
   \item Nalu is installed on the cluster under: \texttt{/shared/codes/nalu/build/Nalu}. 
   The executable is located under: \texttt{/shared/codes/nalu/build/Nalu/build/naluX}
   \item Nightly regression tests can be found in the base \texttt{Nalu} directory, \texttt{Nalu/reg\_tests/test\_files}.
   For example, \texttt{Nalu/reg\_tests/test\_files/ductWedge} is a three-dimensional duct simulation consisting of Wedge6 elements. This input file references a required mesh file, \texttt{nalu/reg\_tests/mesh} and a solver XML file under: \texttt{Nalu/reg\_tests/xml}. A common mistake when running a simulation is copying over input files and not modifying the paths to the mesh and XML file.
    \item Make a directory for relevant simulation files (the input file, 
        the mesh file, and the XML solver settings file). 
    \item Copy the template slurm file from its shared location to your current
        directory: 

        \begin{center}
        \texttt{cp /shared/scripts/nalu.slurm .}
        \end{center}
    \item Make changes to the paths, settings and filenames in the template as
        necessary. Specifically, the default slurm file specifies an \texttt{input.i}, a number of nodes, \texttt{-N}, number mpi ranks, \texttt{-n}, etc. Note that the name of your job is currently set as "nalu-test", and stad output (such as errors) are sent to job.*.out, where * is a unique ID/number.
    \item Submit the job using 

        \begin{center}
        \texttt{sbatch nalu.slurm}
        \end{center}
    \item To view the status of your job, use 

        \begin{center}
        \texttt{squeue}
        \end{center}
        \item A common mistake when submitting a simulation is not changing the \texttt{simulation\_time} or \texttt{termination\_time} line. Therefore, the simulation ends very early, as per the nightly regression test philosophy.
        \item Make sure that the frequency for results output or restart output is "sane", i.e., not every step since you can easily fill the disk partition.
\end{itemize}

\noindent
To run Paraview on the cluster: 
\begin{itemize}[noitemsep]
    \item Log into the cluster using X11 forwarding:

      \begin{center}
      \texttt{ssh -X sunetid@me469-cluster.stanford.edu}
      \end{center}

      \textit{Note:} Depending on what OS your computer uses, you may need to
      install an X11 forwarding tool. For MacOS users, please download 
      XQuartz from \url{https://www.xquartz.org/} -- version number should be 2.8.5. 

    \item Load the installed Paraview module:
      \begin{center}
      \texttt{module load apps/paraview/5.5.0-RC4-Qt5-MPI}
      \end{center}
    
    \item Launch the Paraview GUI using
      \begin{center}
      \texttt{paraview}
      \end{center}
     
    \item The above remote procedure can be very slow to execute commands. Preferably, you can use the Secure Copy utility to 
    copy the results files to your local machine, that requires a local installation of 
        Paraview (see:  \url{https://www.paraview.org/download/}.
      \begin{center}
      \texttt{scp -r sunetid@me469-cluster.stanford.edu:/path/to/paraview/directory ./}
      \end{center}

\item For massively parallel computing visualization, the best path forward is a server/client model where Paraview is launched at both your local machine and the me469 cluster. However, this is not available - at present. As such, you will likely need to routinely move data off of the cluster to visualize the data.
\end{itemize}

\noindent
Cluster support is at \texttt{hpccenter@stanford.edu}, however, if you run into any issues, try to contact me \texttt{spdomin@stanford.edu} first. Please remember that the cluster is a \texttt{shared resource}. It may be easy to take too many nodes at a time, or fill up a disk. Mistakes are expected; simply try to make sure that you are using the cluster in a fair manner (limit jobs when the machine is highly subscribed) and be careful about filling a disk with excessive simulation results.

\end{document}
